\documentclass[10pt,a4paper]{article}
\usepackage[utf8]{inputenc}
\usepackage[spanish]{babel}
\usepackage{times}
\usepackage{hyperref} % Re-enable hyperref
\usepackage{enumitem}
\usepackage{tikz}       % For graphics (might still be useful for other things, or can remove if only for pgf-radar)
% \usepackage{pgf-radar}  % No longer using pgf-radar, using Matplotlib image
\usepackage{amsmath}
\usepackage{amsfonts}
\usepackage{amssymb}
\usepackage{graphicx}
\usepackage{geometry}
\geometry{a4paper, margin=1in}
\usepackage{array}
\usepackage{sectsty}
\usepackage{ragged2e}

% Hyperlink setup - re-enabled
\hypersetup{
   colorlinks=false,
   pdfborder={0 0 0},
   breaklinks=true % Allows links to break across lines
}

\sectionfont{\large\bfseries}
\subsectionfont{\normalsize\bfseries}

\begin{document}

% --- Title and Puntuacion ---
\begin{center}
    {\Huge\bfseries La "demoledora" imagen de Jordi Pujol en el banquillo\par}
    \vspace{0.5em}
    {\large\bfseries Puntuación: 80\par}
\end{center}
\vspace{1em}

% --- Autor, Fuente, Fecha ---
\noindent\textbf{Autor:} Aqu\'{i} Desde El Comienzo. Licenciado En Derecho Y Periodismo. M\'{a}s De Veinte A\~{n}os En La Informaci\'{o}n De Tribunales. He Cubierto La Investigaci\'{o}n Y Los Juicios Del, El, Caso N\'{o}os, Y El, Proc\'{e}s, Entre Otros Muchos. Tambi\'{e}n Pas\'{e} Por Las Secciones De Reportajes Y Madrid. Antes, En Associated Press Y Abc. Autor De, Viaje A Las Fuentes Del Nilo Azul, Y, Hern\'{a}n Cort\'{e}s. Los Pasos Borrados \hfill \textbf{Fuente:} LaRazon \hfill \textbf{Fecha:} 12 de May de 2025\par
\vspace{1em}

% --- Cuerpo de la Noticia ---
\section*{Cuerpo de la Noticia}
\setlength{\parskip}{1em}
\setlength{\parindent}{0em}
En apenas seis meses, el expresidente de la Generalitat Jordi Pujol y sus siete hijos se sentar\'{a}n en el banquillo de la Audiencia Nacional acusados de delitos de asociaci\'{o}n il\'{i}cita, blanqueo y fraude fiscal, entre otros.
\par\medskip
Que la que fuera la familia con mayor poder en Catalu\~{n}a durante m\'{a}s de dos d\'{e}cadas sea juzgada por supuestas actividades corruptas «aprovechando su posici\'{o}n privilegiada de ascendencia en la vida pol\'{i}tica, social y econ\'{o}mica catalana» refuerza presunci\'{o}n de inocencia mediante la confianza en una Justicia igual para todos. Pero lenta, muy lenta.
\par\medskip
Tanto que cuando comience el medi\'{a}tico juicio el pr\'{o}ximo 10 de noviembre Jordi Pujol tendr\'{a} ya 95 a\~{n}os los cumplir\'{a} el 9 de junio y su imagen ante el tribunal como ep\'{i}tome visual de esa proverbial Justicia renqueante -lastrada por la falta de medios y con una ratio de jueces por habitante por debajo de la media europea- preocupa en la Audiencia Nacional, donde las fuentes consultadas no dudan en calificarla de «demoledora».
\par\medskip
No en vano, han pasado m\'{a}s de doce a\~{n}os desde que, en enero de 2013, Victoria \'{A}lvarez, exnovia de Jordi Pujol Ferrusola, primog\'{e}nito del l\'{i}der convergente, denunciase en la Audiencia Nacional tras haberlo hecho ante la Polic\'{i}a en diciembre del a\~{n}o anterior los viajes a Andorra con bolsas de dinero y las cuentas familiares ocultas, que Jordi Pujol acabar\'{i}a reconociendo en julio de 2014, aunque atribuy\'{e}ndolas a una herencia paterna.
\par\medskip
Ese 17 de enero en el que el juez Pablo Ruz, primer instructor de la causa, tom\'{o} declaraci\'{o}n a rega\~{n}adientes a Victoria \'{A}lvarez obligado por la Secci\'{o}n Tercera de la Sala de lo Penal a instancias de la Fiscal\'{i}a, el futuro de las diligencias se adivinaba incierto.
\par\medskip
En los pasillos de la Audiencia Nacional en la madrile\~{n}a calle Prim la sede de los juzgados de instrucci\'{o}n estaba en obras se escuchaba incluso que la expareja de Pujol Ferrusolaactuaba por despecho y que, tras las necesarias actuaciones de investigaci\'{o}n, sus acusaciones estaban abocadas al archivo.
\par\medskip
"?`Qu\'{e} co\~{n}o es eso de la UDEF?"
\par\medskip
Pero no. La investigaci\'{o}n sali\'{o} adelante impulsada por los informes de la Unidad de Delincuencia Econ\'{o}mica y Fiscal de la Polic\'{i}a  «?`qu\'{e} co\~{n}o es eso de la UDEF?», se pregunt\'{o} Jordi Pujol en el Parlament meses despu\'{e}s como si la cosa no fuera con \'{e}l. Durante m\'{a}s de una d\'{e}cada, al juez Pablo Ruz le siguieron otros dos: Jos\'{e} de la Mata y Santiago Pedraz.
\par\medskip
Una investigaci\'{o}n interminable casi todas en las que hay que librar comisiones rogatorias al extranjero reclamando auxilio judicial internacional lo son que se vio prolongada incluso m\'{a}s a\'{u}n si cabe durante m\'{a}s de dos a\~{n}os por el cribado del sumario de datos privados de la familia Pujol y las recurrentes quejas de su defensa sobre los problemas inform\'{a}ticos para acceder a las diligencias para poder realizar ese expurgo que no dejara a la intemperie correos e im\'{a}genes de su intimidad personal.
\par\medskip
La prolongada tramitaci\'{o}n de la causa durante la cual fue apartada del procedimiento por padecer alzh\'{e}imer la esposa de Jordi Pujol, Marta Ferrusola, fallecida en julio del pasado a\~{n}o acarrear\'{a}, seg\'{u}n las fuentes consultadas, que el tribunal est\'{e} obligado a aplicar una atenuante muy cualificada de dilaciones indebidas, lo que en caso de condena la Fiscal\'{i}a Anticorrupci\'{o}n reclama para el expresidente de la Generalitat una pena de nueve a\~{n}os de prisi\'{o}n por asociaci\'{o}n il\'{i}cita y blanqueo de capitales esta se ver\'{i}a sensiblemente reducida.
\par\medskip
Una circunstancia que, apuntan, avivar\'{i}a a\'{u}n m\'{a}s la perplejidad ciudadana ante las consecuencias de una respuesta judicial tan dilatada en el tiempo.
\par\medskip
"?`Qui\'{e}n cree que a un anciano lo 'rehabilitas'?"
\par\medskip
«La Justicia tard\'{i}a no es justicia», apuntan fuentes jur\'{i}dicas, que resaltan que as\'{i} «se diluye el efecto preventivo general de la pena: la coacci\'{o}n psicol\'{o}gica a la colectividad para evitar delitos futuros».
\par\medskip
En todo caso, a\~{n}aden, «juzgar a un anciano de 95 a\~{n}os casa mal con los fundamentos mismos de la Justicia penal» en la medida en que «se desdibuja el fundamento reeducador y rehabilitador de la pena».
\par\medskip
«Un anciano no se defiende, es complaciente y tiene la guardia baja», insisten. «Interrogarle es casi un abuso, porque no miden el alcance de lo que dicen».
\par\medskip
«?`Qui\'{e}n cree que a un anciano lo rehabilitas?», se preguntan al respecto antes de recordar que nuestra Constituci\'{o}n «expresa la resocializaci\'{o}n como fundamento y finalidad de la pena», lo que en este caso resaltan «se percibe como una inutilidad».
\par\medskip
Frente a este posicionamiento, el de quienes defienden que, al margen de su edad, Jordi Pujol debe rendir cuentas ante la Justicia por los supuestos hechos delictivos que se le atribuyen.
\par\medskip
De manera que una instrucci\'{o}n desmesurada no puede servir al expresident de parapeto, por su avanzada edad, para eludir el juicio.
\par\medskip
En cualquier caso, su presencia en la sala puede ser fugaz (est\'{a} previsto que el juicio concluya en abril de 2026) dado que se da por hecho que, llegado el caso, su defensa solicitar\'{a} al tribunal que se le exima de acudir tras su declaraci\'{o}n hasta el tr\'{a}mite de \'{u}ltima palabra.
\par\medskip
Siempre y cuando, claro, sus abogados no aporten antes un certificado m\'{e}dico que constate un posible deterioro cognitivo, en virtud del cual soliciten a la Sala el archivo de las actuaciones.
\par\medskip
En la retina de la Audiencia Nacional, otra imagen que en 2009 escenific\'{o} la «pena de banquillo». Dos exdirigentes de CiU muy pr\'{o}ximos al propio Pujol, Maci\`{a} Alavedra y Llu\'{i}s Prenafeta, fueron fotografiados a las puertas del tribunal esposados y sujetando bolsas de basura con sus pertenencias tras ser trasladados desde la c\'{a}rcel de Soto del Real para declarar ante el juez Baltasar Garz\'{o}n por la «operaci\'{o}n Pretoria».
\par\medskip
La investigaci\'{o}n interna fue archivada, pero el revuelo originado oblig\'{o} a revisar los protocolos de traslados de detenidos para evitar que una situaci\'{o}n as\'{i} se repitiese.
\vspace{1em}

% --- URL (Re-enabled and moved) ---
\section*{URL}
\url{ https://www.larazon.es/espana/demoledora-imagen-jordi-pujol-banquillo\_20250512681e4a02f7f20a10d038cb17.html }
\vspace{1em}

% --- Resumen de la Valoración ---
\section*{Resumen de la Valoración}
    La noticia es impactante y bien estructurada, con buen contexto y an\'{a}lisis cr\'{i}tico, pero podr\'{i}a mejorar en diversidad de fuentes y contexto sociopol\'{i}tico.
\vspace{1em}

% --- Resumen de la Valoración del Titular ---
\section*{Resumen de la Valoración del Titular}
    Titular con baja objetividad, alto componente clickbait, impreciso y con uso sensacionalista del lenguaje period\'{i}stico.
\vspace{1em}

% --- Spider Chart for Puntuacion Individual (from Matplotlib image) ---
\clearpage
\section*{Gráfico de Puntuaciones Individuales}
\begin{figure}[h!] % [h!] tries to place it "here"
   \centering
   \includegraphics[width=0.8\textwidth, keepaspectratio]{spider_chart.png}
   % \caption{Gráfico de Puntuaciones Individuales.} % Optional caption
\end{figure}
\clearpage % Add a page break after the figure, if desired, to ensure it doesn't overlap

% --- Valoracion General (String Summary) ---
\section*{Valoración General}
    La noticia sobre Jordi Pujol y su familia, acusados de delitos como asociaci\'{o}n il\'{i}cita, blanqueo y fraude fiscal, destaca por su impacto en la percepci\'{o}n p\'{u}blica debido a la notoriedad de los involucrados. Presenta un relato bien estructurado, respaldado por fuentes y datos cronol\'{o}gicos, que permite comprender la dimensi\'{o}n del caso y sus implicaciones en el sistema judicial espa\~{n}ol.
\par\medskip
\textbf{Fortalezas:}  - Proporciona un contexto hist\'{o}rico detallado, desde la denuncia inicial en 2013 hasta el juicio programado para noviembre, resaltando el papel de la UDEF en la investigaci\'{o}n. - Ofrece un an\'{a}lisis cr\'{i}tico sobre la lentitud del sistema judicial y su impacto en la eficacia punitiva, aportando perspectiva a las dilaciones indebidas y su efecto en la percepci\'{o}n p\'{u}blica.
\par\medskip
\textbf{\'{A}reas de Mejora:}  - Podr\'{i}a beneficiarse de m\'{a}s declaraciones directas de los actores clave, enriqueciendo as\'{i} el contenido con perspectivas diversas y m\'{a}s detalladas. - Explorar con mayor profundidad el contexto pol\'{i}tico y social del caso y sus repercusiones en Catalu\~{n}a y toda Espa\~{n}a dar\'{i}a una visi\'{o}n m\'{a}s hol\'{i}stica del impacto del juicio. - Ser\'{i}a valioso abordar las implicaciones \'{e}ticas y legales de juzgar a una persona de avanzada edad, lo que podr\'{i}a ofrecer una narrativa m\'{a}s equilibrada y amplia.
\par\medskip
En resumen, la noticia ofrece un reportaje detallado y cr\'{i}tico, con la posibilidad de mejorar integrando m\'{a}s voces y an\'{a}lisis sobre el contexto sociopol\'{i}tico y las dimensiones humanas del juicio.
\vspace{1em}

% --- Valoración del Titular ---
\section*{Valoración del Titular}
        \textbf{Análisis del titular:} \\ 
        An\'{a}lisis del titular: "La 'demoledora' imagen de Jordi Pujol en el banquillo"
\par\medskip
1. \textbf{Veracidad}:    - El uso del t\'{e}rmino "demoledora" sugiere una imagen con un impacto significativo o negativo sobre Jordi Pujol, pero no ofrece detalles concretos sobre en qu\'{e} consiste dicho impacto. Esta vaguedad puede inducir a la especulaci\'{o}n y no proporciona informaci\'{o}n precisa sobre el evento o la imagen en s\'{i}, lo que afecta la veracidad.
\par\medskip
2. \textbf{Claridad}:    - El titular carece de especificidad al no detallar lo que hace que la imagen sea "demoledora". La falta de informaci\'{o}n concreta impide al lector entender inmediatamente el contexto de la noticia, provocando confusi\'{o}n sobre la naturaleza e implicaciones de la imagen.
\par\medskip
3. \textbf{Relevancia}:    - Aunque el tema es de relevancia judicial importante, dado el contexto de Jordi Pujol en el banquillo, el titular emplea un enfoque sensacionalista, lo que puede distraer de los aspectos esenciales del caso. Este tratamiento puede restar atenci\'{o}n a los detalles relevantes del proceso judicial.
\par\medskip
4. \textbf{Clickbait}:    - El uso de comillas y el calificativo "demoledora" est\'{a}n destinados a generar una respuesta emocional y atraer clics, priorizando el impacto emocional sobre la informaci\'{o}n objetiva y directa.
\par\medskip
\textbf{Puntuaci\'{o}n}: - \textbf{Clickbait}: 7/10    - El titular emplea un lenguaje sensacionalista dise\~{n}ado para captar la atenci\'{o}n del lector, m\'{a}s que para ofrecer una informaci\'{o}n clara y objetiva. - \textbf{Objetividad informativa}: 3/10    - La falta de claridad y concreci\'{o}n limita el valor informativo del titular, aline\'{a}ndolo m\'{a}s con pr\'{a}cticas t\'{i}picas de clickbait.
\par\medskip
\textbf{Conclusi\'{o}n}: El titular est\'{a} dise\~{n}ado para atraer atenci\'{o}n mediante el uso de un lenguaje emocionalmente cargado y ambivalente, m\'{a}s que para informar de manera clara y objetiva. Aunque el tema tiene relevancia, el estilo empleado lo convierte en un ejemplo de clickbait, poniendo el \'{e}nfasis en la emoci\'{o}n y el impacto en lugar de en la veracidad y claridad.
\par\medskip
\textbf{T\'{I}TULO PROPUESTO}: Jordi Pujol comparece ante el tribunal por cargos judiciales.
        \vspace{1em}
\vspace{1em}

% --- Texto Referencia (Consolidated) ---
\section*{Texto de Referencia}
    \textit{(Mostrando desde el campo de diccionario directo: texto\_referencia\_diccionario)}
    \begin{itemize}[leftmargin=*]
            \item \textbf{Referencia El art\'{i}culo emplea t\'{e}rminos como "demoledora" para describir la imagen de Pujol en el juicio, y "Justicia renqueante" al hablar del sistema judicial\_:} 1
            \item \textbf{Referencia Se identifican tres opiniones expl\'{i}citas: "La Justicia tard\'{i}a no es justicia", "juzgar a un anciano de 95 a\~{n}os casa mal con los fundamentos mismos de la Justicia penal" y "Un anciano no se defiende, es complaciente y tiene la guardia baja"\_:} 2
            \item \textbf{Referencia La noticia referencia m\'{u}ltiples fuentes, tales como "fuentes consultadas en la Audiencia Nacional", "fuentes jur\'{i}dicas", Victoria \'{A}lvarez (exnovia de Jordi Pujol Ferrusola) y la Fiscal\'{i}a Anticorrupci\'{o}n\_:} 3
            \item \textbf{Referencia Aunque la noticia menciona "fuentes consultadas" o "fuentes jur\'{i}dicas", no las identifica concretamente\_:} 4
            \item \textbf{Referencia La imagen de Jordi Pujol, ex-presidente de la Generalitat de Catalu\~{n}a, sentado en el banquillo de los acusados, ha impactado significativamente en el entorno pol\'{i}tico y social de Espa\~{n}a\_:} 5
            \item \textbf{Referencia La noticia sobre Jordi Pujol ofrece una base s\'{o}lida al proporcionar una cronolog\'{i}a detallada del caso y abordar la evoluci\'{o}n de la investigaci\'{o}n judicial\_:} 6
            \item \textbf{Referencia El uso de un lenguaje preciso y descriptivo ayuda a desentra\~{n}ar conceptos complejos como "comisiones rogatorias" y "dilaciones indebidas"\_:} 7
            \item \textbf{Referencia Su enfoque se destaca por incluir un an\'{a}lisis cr\'{i}tico sobre la lentitud del sistema judicial espa\~{n}ol y reflexiones sobre los fundamentos de la justicia penal\_:} 8
            \item \textbf{Referencia La noticia presenta un contexto hist\'{o}rico s\'{o}lido, comenzando con la denuncia inicial en 2013 y cubriendo eventos importantes como la confesi\'{o}n de Jordi Pujol sobre sus cuentas en 2014\_:} 9
            \item \textbf{Referencia La noticia logra un equilibrio entre informar sobre un caso de alto perfil y mantener el respeto por la privacidad de los involucrados\_:} 10
    \end{itemize}
\vspace{1em}

% --- Valoraciones (Dictionary of AI interpretations) ---
% This section is to be removed as per user request.
% \section*{Valoraciones (Interpretaciones IA)}
% %    ... (content previously here) ...
% % \vspace{1em}

\end{document}
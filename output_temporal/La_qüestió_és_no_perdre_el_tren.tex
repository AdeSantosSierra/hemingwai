\documentclass[10pt,a4paper]{article}
\usepackage[utf8]{inputenc}
\usepackage[spanish]{babel}
\usepackage{times}
\usepackage{hyperref} % Re-enable hyperref
\usepackage{enumitem}
\usepackage{tikz}       % For graphics (might still be useful for other things, or can remove if only for pgf-radar)
% \usepackage{pgf-radar}  % No longer using pgf-radar, using Matplotlib image
\usepackage{amsmath}
\usepackage{amsfonts}
\usepackage{amssymb}
\usepackage{graphicx}
\usepackage{geometry}
\geometry{a4paper, margin=1in}
\usepackage{array}
\usepackage{sectsty}
\usepackage{ragged2e}

% Hyperlink setup - re-enabled
\hypersetup{
   colorlinks=false,
   pdfborder={0 0 0},
   breaklinks=true % Allows links to break across lines
}

\sectionfont{\large\bfseries}
\subsectionfont{\normalsize\bfseries}

\begin{document}

% --- Title and Puntuacion ---
\begin{center}
    {\Huge\bfseries La q\"{u}esti\'{o} \'{e}s no perdre el tren\par}
    \vspace{0.5em}
    {\large\bfseries Puntuación: 83\par}
\end{center}
\vspace{1em}

% --- Autor, Fuente, Fecha ---
\noindent\textbf{Autor:} Josep Cartany\`{a} \hfill \textbf{Fuente:} Fuente no identificada \hfill \textbf{Fecha:} N/A\par
\vspace{1em}

% --- Cuerpo de la Noticia ---
\section*{Cuerpo de la Noticia}
\setlength{\parskip}{1em}
\setlength{\parindent}{0em}
Diuen que hi ha trens que passen poques vegades a la vida i nhi ha que nom\'{e}s saturen un cop. Aquesta podria ser la sensaci\'{o} que viu actualment Tarragona dil·lusi\'{o} i dangoixa a la vegada per agafar els trens que la portaran als canvis i a la transformaci\'{o} de la ciutat.
\par\medskip
Lentrada en vigor del nou contracte de la brossa \'{e}s un daquells trens que circulen amb retard per\`{o} que saps que tard o dhora arribaran. El mes passat, el TSJC va rebutjar les mesures cautelars sol·licitades per lempresa Paprec (primera classificada del concurs) contra la resoluci\'{o} del Tribunal Catal\`{a} de Contractes del Sector P\'{u}blic, que deia que el contracte de la brossa sha dadjudicar a Urbaser (segona classificada). Lalcalde de Tarragona, Rub\'{e}n Vi\~{n}uales (PSC), celebrava la decisi\'{o} judicial. Ara, lequip de govern est\`{a} en espera duna resoluci\'{o} en el mateix sentit del jutjat contenci\'{o}s administratiu per adjudicar, finalment, el contracte a Urbaser. La bona not\'{i}cia \'{e}s que estem molt m\'{e}s a prop de solucionar-ho, deia el batlle.
\par\medskip
I mentrestant, qu\`{e}? A ning\'{u} se li escapa que el debat sobre la manca de neteja existeix en qualsevol barri de la ciutat. Tant la Federaci\'{o} dAssociacions de Ve\"{i}ns de Tarragona (FAVT) com alguns grups de loposici\'{o} reclamen a Vi\~{n}uales solucions provisionals mentre no entri en vigor el nou contracte: lloguer de maquin\`{a}ria, actuacions durg\`{e}ncia, neteges localitzades i m\'{e}s inspecci\'{o} del servei que presta lencara empresa adjudicat\`{a}ria actual, FCC. Lalcalde demana paci\`{e}ncia i recorda que nom\'{e}s shan pogut externalitzar aquelles coses que no estan incloses en lactual contracte com la neteja de les males herbes o la deixalleria.
\par\medskip
A Tarragona no para dincrementar-se l\'{u}s del bus urb\`{a}, amb 11 milions de viatgers anuals. A priori , aquesta hauria de ser una bona not\'{i}cia per a la sostenibilitat, per\`{o} la realitat \'{e}s que larbre no ens deixa veure el bosc. Quan els ciutadans agafen qualsevol bus en hores punta, la majoria de busos van col·lapsats. \'{E}s per aix\`{o} que la FAVT adverteix que una reducci\'{o} de freq\"{u}\`{e}ncies seria inacceptable i considera imprescindible no nom\'{e}s renovar la flota, sin\'{o} tamb\'{e} ampliar-la per atendre una demanda que no para de cr\'{e}ixer.
\par\medskip
Hi ha trens, per\`{o}, que valdria m\'{e}s que passessin lluny del nucli urb\`{a} de la ciutat. S\'{o}n els que porten mercaderies perilloses. Una reivindicaci\'{o} de la Plataforma Mercaderies per lInterior, que ha aconseguit aglutinar el m\'{o}n ve\"{i}nal, ecologista i pol\'{i}tic. La plataforma recorda que ha proposat alternatives per evitar que les mercaderies perilloses circulin pel mig de les ciutats, com passa a Tarragona. A finals del 2024, el ministeri va presentar 15 alternatives a partir duna reclamaci\'{o} feta per lassociaci\'{o} i els ajuntaments del territori. Sembla que el govern espanyol aposti per una, que passa per la construcci\'{o} dun llarg viaducte per sobre del Francol\'{i} i un segon viaducte que passi per damunt de lactual l\'{i}nia dalta velocitat, a laltura de la Secuita. Nosaltres defensem no fer el viaducte del Francol\'{i}, sin\'{o} passar per la dreta de la l\'{i}nia de lEuromed i seguir en paral·lel a lactual l\'{i}nia de lAVE fins a Roda de Ber\`{a}, defensa la plataforma.
\par\medskip
Per\`{o} el 28 de gener passat un contratemps inesperat va trencar el clima de consens al voltant daquesta q\"{u}esti\'{o}. Leurodiputada Sandra G\'{o}mez (PSOE) va votar en contra de la reclamaci\'{o} presentada per Mercaderies per lInterior al Parlament Europeu de desviar els trens de mercaderies per la via entre Reus i Roda de Ber\`{a}. Daquesta manera, els socialistes europeus es desmarcaven de la posici\'{o} del PSC i dels alcaldes del territori, que aposten per allunyar les mercaderies de la costa. Vi\~{n}uales avisava el mateix dia que no es mouran de la seva posici\'{o} perqu\`{e} t\'{e} un ampl\'{i}ssim consens.
\par\medskip
No vols caldo? Doncs dues tasses. El ple de Tarragona de febrer va aprovar per unanimitat portar al jutjat la Comissi\'{o} de Territori de Catalunya per un canvi del pla director de la gran ind\'{u}stria que implicaria el pas de 46.000 camions per dins de la ciutat, 8.000 dels quals amb mercaderies perilloses. La modificaci\'{o} preveu eliminar lestaci\'{o} intermodal que lempresa BASF volia fer als seus terrenys, que finalment ha desistit. Aix\`{o} implicaria una reconfiguraci\'{o} del tr\`{a}nsit de mercaderies, ja que se centralitzarien a la terminal de la Boella del Port de Tarragona, i faria que els camions sortissin cap a lA-27, creuessin pol\'{i}gons i passessin pel costat dels barris de Ponent. Sembla, per\`{o}, que el secretari general de Mobilitat es mostra obert a trobar una soluci\'{o} dialogada.
\vspace{1em}

% --- URL (Re-enabled and moved) ---
\section*{URL}
\url{ https://www.elpuntavui.cat/politica/article/17-politica/2513748-la-queestio-es-no-perdre-el-tren.html }
\vspace{1em}

% --- Resumen de la Valoración ---
\section*{Resumen de la Valoración}
    La noticia destaca por su an\'{a}lisis detallado, uso de m\'{u}ltiples fuentes cre\'{i}bles y enfoque equilibrado, aunque carece de datos econ\'{o}micos espec\'{i}ficos.
\vspace{1em}

% --- Resumen de la Valoración del Titular ---
\section*{Resumen de la Valoración del Titular}
    Titular coherente y de calidad period\'{i}stica, con alta puntuaci\'{o}n en claridad, veracidad y relevancia, y m\'{i}nimo nivel de clickbait.
\vspace{1em}

% --- Spider Chart for Puntuacion Individual (from Matplotlib image) ---
\clearpage
\section*{Gráfico de Puntuaciones Individuales}
\begin{figure}[h!] % [h!] tries to place it "here"
   \centering
   \includegraphics[width=0.8\textwidth, keepaspectratio]{spider_chart.png}
   % \caption{Gráfico de Puntuaciones Individuales.} % Optional caption
\end{figure}
\clearpage % Add a page break after the figure, if desired, to ensure it doesn't overlap

% --- Valoracion General (String Summary) ---
\section*{Valoración General}
    La noticia "La q\"{u}esti\'{o} \'{e}s no perdre el tren" ofrece un an\'{a}lisis detallado de varios desaf\'{i}os urbanos cr\'{i}ticos que enfrenta Tarragona, como la gesti\'{o}n de residuos, el transporte p\'{u}blico y el tr\'{a}nsito de mercanc\'{i}as peligrosas. Se destaca por el uso de m\'{u}ltiples fuentes que aportan credibilidad y un enfoque equilibrado al incluir diversas perspectivas, como la del alcalde, asociaciones vecinales y plataformas pol\'{i}ticas. Sin embargo, hay \'{a}reas que podr\'{i}an enriquecerse, como la inclusi\'{o}n de datos econ\'{o}micos detallados y la exploraci\'{o}n de alternativas espec\'{i}ficas para los problemas tratados. A pesar de estos aspectos, la noticia es considerada de buena calidad por su claridad y cobertura exhaustiva, proporcionando un an\'{a}lisis robusto que logra captar la atenci\'{o}n del lector a trav\'{e}s de una narrativa accesible y un contexto bien desarrollado.
\vspace{1em}

% --- Valoración del Titular ---
\section*{Valoración del Titular}
        \textbf{Análisis del titular:} \\ 
        El titular "La q\"{u}esti\'{o} \'{e}s no perdre el tren" sigue siendo evaluado positivamente, atendiendo a los cuatro criterios fundamentales que garantizan su calidad: veracidad, claridad, relevancia y nivel de clickbait. A continuaci\'{o}n, se ofrece un an\'{a}lisis m\'{a}s detallado:
\par\medskip
1. \textbf{Veracidad}: Este titular ofrece una met\'{a}fora clara y accesible. No intenta enga\~{n}ar sino transmitir un concepto com\'{u}n a trav\'{e}s de un lenguaje coloquial que resuena con el p\'{u}blico. Debido a su enfoque transparente y no literal, recibe una puntuaci\'{o}n s\'{o}lida de 8/10 en veracidad.
\par\medskip
2. \textbf{Claridad}: La claridad es una de sus fortalezas. El juego de palabras es evidente y el mensaje se comunica de manera sencilla y efectiva. Esta facilidad de comprensi\'{o}n es vital en los titulares, asegur\'{a}ndole un merecido 9/10 en este aspecto.
\par\medskip
3. \textbf{Relevancia}: La met\'{a}fora de "no perder el tren" es una expresi\'{o}n familiar que alude a la urgencia de no desaprovechar oportunidades. Es aplicable a diversos contextos, lo que le otorga una sensaci\'{o}n de pertinencia y motivaci\'{o}n con una puntuaci\'{o}n de 8/10.
\par\medskip
4. \textbf{Nivel de Clickbait}: El titular evita t\'{a}cticas de clickbait. No exagera ni induce al error para capturar la atenci\'{o}n del lector, estableciendo expectativas realistas. Esto se refleja en su baja puntuaci\'{o}n de 2/10 en clickbait, lo que confirma su integridad.
\par\medskip
\textbf{Evaluaci\'{o}n global}: Con una puntuaci\'{o}n final de 8.5/10, el titular "La q\"{u}esti\'{o} \'{e}s no perdre el tren" es valorado como un ejemplo de titular bien logrado: claro, relevante, veraz y libre de manipulaciones. 
\par\medskip
Ya que, seg\'{u}n la evaluaci\'{o}n actual, no se considera que el titular sea clickbait, no se propone una versi\'{o}n alternativa. Sin embargo, si en alg\'{u}n contexto espec\'{i}fico se buscara mayor literalidad, podr\'{i}a emplearse:
\par\medskip
TITULO PROPUESTO: "La importancia de no dejar pasar las oportunidades"
        \vspace{1em}
        % Titular reformulado eliminado según petición del usuario
\vspace{1em}

% --- Texto Referencia (Consolidated) ---
\section*{Texto de Referencia}
    \textit{(Mostrando desde el campo de diccionario directo: texto\_referencia\_diccionario)}
    \begin{itemize}[leftmargin=*]
            \item \textbf{Referencia Diuen que hi ha trens que passen poques vegades a la vida i nhi ha que nom\'{e}s saturen un cop\_ Aquesta podria ser la sensaci\'{o} que viu actualment Tarragona dil·lusi\'{o} i dangoixa a la vegada per agafar els trens que la portaran als canvis i a la transformaci\'{o} de la ciutat\_:} \{'1': 'La noticia ofrece una descripci\'{o}n general adecuada de la situaci\'{o}n en Tarragona, centr\'{a}ndose en datos y declaraciones relevantes.
            \item \textbf{Referencia Tant la Federaci\'{o} dAssociacions de Ve\"{i}ns de Tarragona (FAVT) com alguns grups de loposici\'{o} reclamen a Vi\~{n}uales solucions provisionals mentre no entri en vigor el nou contracte\_:} \{'1': 'La noticia ofrece una descripci\'{o}n general adecuada de la situaci\'{o}n en Tarragona, centr\'{a}ndose en datos y declaraciones relevantes.
            \item \textbf{Referencia Lentrada en vigor del nou contracte de la brossa \'{e}s un daquells trens que circulen amb retard per\`{o} que saps que tard o dhora arribaran\_:} \{'1': 'La noticia ofrece una descripci\'{o}n general adecuada de la situaci\'{o}n en Tarragona, centr\'{a}ndose en datos y declaraciones relevantes.
            \item \textbf{Referencia A priori , aquesta hauria de ser una bona not\'{i}cia per a la sostenibilitat, per\`{o} la realitat \'{e}s que larbre no ens deixa veure el bosc\_:} \{'1': 'La noticia ofrece una descripci\'{o}n general adecuada de la situaci\'{o}n en Tarragona, centr\'{a}ndose en datos y declaraciones relevantes.
            \item \textbf{Referencia lalcalde demana paci\`{e}ncia i recorda que nom\'{e}s shan pogut externalitzar aquelles coses que no estan incloses en lactual contracte com la neteja de les males herbes o la deixalleria\_:} \{'1': 'La noticia ofrece una descripci\'{o}n general adecuada de la situaci\'{o}n en Tarragona, centr\'{a}ndose en datos y declaraciones relevantes.
    \end{itemize}
\vspace{1em}

% --- Valoraciones (Dictionary of AI interpretations) ---
% This section is to be removed as per user request.
% \section*{Valoraciones (Interpretaciones IA)}
% %    ... (content previously here) ...
% % \vspace{1em}

% --- Análisis de Verificación de Hechos ---
\section*{Análisis de Verificación de Hechos}
    La noticia contiene principalmente afirmaciones sobre el proceso del nuevo contrato de la recogida de residuos en Tarragona, el debate sobre la limpieza municipal, el aumento de usuarios de bus urbano, y el transporte de mercanc\'{i}as peligrosas por el entorno urbano. A continuaci\'{o}n, verifico puntualmente cada dato referido y se\~{n}alo su adecuaci\'{o}n seg\'{u}n fuentes institucionales y medios de referencia.
\par\medskip
Sobre el contrato de la brossa:
 Es cierto que el nuevo contrato de recogida de basura en Tarragona ha estado marcado por recursos y retrasos. El Tribunal Superior de Just\'{i}cia de Catalunya desestim\'{o} en febrero de 2025 las medidas cautelares solicitadas por Paprec contra la resoluci\'{o}n del Tribunal Catal\`{a} de Contractes del Sector P\'{u}blic, que permit\'{i}a adjudicar el contrato a Urbaser, la segunda clasificada, tras la eliminaci\'{o}n de Paprec por cuestiones t\'{e}cnicas y administrativas. Esta cronolog\'{i}a coincide con lo referido en la noticia[3][4].
\par\medskip
 La empresa Urbaser ha sido finalmente la adjudicataria, tras aval judicial y pol\'{i}tico, en un proceso al que s\'{o}lo le restaba el tr\'{a}mite del acuerdo plenario y cuya firma se preve\'{i}a entre octubre y noviembre de 2025[1][2][3][4][5][8].
\par\medskip
 El contrato tiene una duraci\'{o}n de diez a\~{n}os y un valor de algo m\'{a}s de 20 millones de euros anuales, adem\'{a}s de inversiones adicionales por unos 26 millones de euros, seg\'{u}n comunican medios y el propio Ayuntamiento[1][5][6][8]. El dato coincide con lo indicado en la noticia (el importe no se menciona en el fragmento, pero s\'{i} el proceso y actores implicados).
\par\medskip
 Es exacto que, durante la espera por la entrada en vigor, se han hecho demandas de medidas provisionales como alquiler de maquinaria o refuerzo de inspecci\'{o}n por parte de la Federaci\'{o} dAssociacions de Ve\"{i}ns de Tarragona y la oposici\'{o}n, ante la percepci\'{o}n ciudadana de insuficiencia en la limpieza y la continuidad a\'{u}n de FCC como adjudicataria[1][2][3]. El porcentaje del 80\% de quejas ciudadanas referidas a limpieza, citado en otras declaraciones de Vi\~{n}uales, es comunicado por el propio Ayuntamiento y recogido por la prensa local[1].
\par\medskip
 El alcalde Rub\'{e}n Vi\~{n}uales mantiene el mismo discurso sobre la necesidad de paciencia y ha insistido en que s\'{o}lo pueden externalizarse servicios no cubiertos por FCC, como la limpieza de hierbas o el servicio de deixalleria[1][2].
\par\medskip
Sobre el bus urbano:
 El dato de 11 millones de usuarios anuales en el bus urbano de Tarragona es ver\'{i}dico y reconocido por fuentes oficiales y prensa local a finales de 2024 e inicios de 2025. Por ejemplo, en diciembre de 2024 y enero de 2025 la cifra era repetida por Transports Metropolitans de Tarragona y medios locales. Por tanto, la noticia presenta el dato de manera adecuada.
\par\medskip
 Es correcto que la FAVT reclama no reducir frecuencias y considera imprescindible ampliar y renovar la flota dado el aumento de demanda. Estas demandas han sido recogidas por medios locales a lo largo de 2024 y 2025, coincidiendo con la interpretaci\'{o}n de la noticia.
\par\medskip
Sobre los trenes de mercanc\'{i}as peligrosas y la plataforma Mercaderies per lInterior:
 La noticia se\~{n}ala correctamente el activismo de la plataforma, incluyendo la propuesta de alternativas al paso de mercanc\'{i}as peligrosas por zonas urbanas de Tarragona, hecho confirmado de modo reiterado por la plataforma, medios y ayuntamientos en 2024 y 2025.
\par\medskip
 Es cierto que, a finales de 2024, el Ministerio de Transportes espa\~{n}ol present\'{o} quince alternativas t\'{e}cnicas como respuesta a reclamaciones locales y municipales. Entre ellas, la opci\'{o}n de un viaducto sobre el Francol\'{i} y un segundo sobre la l\'{i}nea de alta velocidad a la altura de la Secuita resulta ser la preferida del Ministerio, como recogen declaraciones oficiales y cobertura medi\'{a}tica especializada en infraestructuras durante 2025.
\par\medskip
 Tambi\'{e}n es correcto que la plataforma defiende otra alternativa, sin viaducto sobre el Francol\'{i}, y siguiendo en paralelo a la actual l\'{i}nea del AVE hasta Roda de Ber\`{a}. Esta postura ha sido abundantemente citada en prensa.
\par\medskip
Sobre la votaci\'{o}n europea y la postura socialista:
 Es cierto que el 28 de enero de 2025, la eurodiputada Sandra G\'{o}mez vot\'{o} en contra de la reclamaci\'{o}n espec\'{i}fica ante el Parlamento Europeo para desviar trenes por la v\'{i}a ReusRoda de Ber\`{a}, desmarc\'{a}ndose as\'{i} de la posici\'{o}n oficial del PSC y alcaldes del territorio, alineados con la postura de la plataforma[las agendas de votaciones europeas y las notas de prensa institucionales recogen esta discrepancia]. El propio alcalde Vi\~{n}uales expres\'{o} p\'{u}blicamente su desacuerdo y mantuvo la voluntad de consenso local.
\par\medskip
Sobre el paso de camiones y la industria:
 Es adecuado el dato de que el pleno municipal aprob\'{o} en febrero de 2025, por unanimidad, recurrir a la justicia la decisi\'{o}n de la Comissi\'{o} de Territori de Catalunya sobre la modificaci\'{o}n del plan director industrial, que autorizar\'{i}a el tr\'{a}nsito de 46.000 camiones anuales (8.000 con mercanc\'{i}as peligrosas) a trav\'{e}s de la ciudad. Las cifras y el origen de la modificaci\'{o}n son exactas seg\'{u}n actas municipales y grandes medios regionales.
\par\medskip
 Tambi\'{e}n es correcta la consecuencia de la renuncia de BASF a la estaci\'{o}n intermodal (desistimiento comunicado p\'{u}blicamente en 2025): el tr\'{a}fico de mercanc\'{i}as se centraliza en la terminal de La Boella, incrementando el paso de camiones por la A-27 y junto a los barrios de Ponent. El secretario general de Mobilitat se ha mostrado dispuesto al di\'{a}logo, seg\'{u}n cobertura period\'{i}stica y declaraciones institucionales de primavera y verano de 2025.
\par\medskip
En conjunto, la noticia reproduce de modo fiel los hechos comprobables principales, datos cuantitativos concretos, nombres y secuencias administrativas reflejadas en los medios de referencia y los portales institucionales en 2025. No se detectan inexactitudes sustantivas, actualizaciones contradictorias ni errores de cifras relevantes; la presentaci\'{o}n secuenciada coincide con el desarrollo real de los acontecimientos, sin alteraci\'{o}n significativa del sentido ni del contexto de las intervenciones institucionales o de las partes implicadas.
\vspace{1em}

\end{document}
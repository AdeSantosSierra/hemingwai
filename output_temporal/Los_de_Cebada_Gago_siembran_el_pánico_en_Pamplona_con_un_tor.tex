\documentclass[10pt,a4paper]{article}
\usepackage[utf8]{inputenc}
\usepackage[spanish]{babel}
\usepackage{times}
\usepackage{hyperref} % Re-enable hyperref
\usepackage{enumitem}
\usepackage{tikz}       % For graphics (might still be useful for other things, or can remove if only for pgf-radar)
% \usepackage{pgf-radar}  % No longer using pgf-radar, using Matplotlib image
\usepackage{amsmath}
\usepackage{amsfonts}
\usepackage{amssymb}
\usepackage{graphicx}
\usepackage{geometry}
\geometry{a4paper, margin=1in}
\usepackage{array}
\usepackage{sectsty}
\usepackage{ragged2e}

% Hyperlink setup - re-enabled
\hypersetup{
   colorlinks=false,
   pdfborder={0 0 0},
   breaklinks=true % Allows links to break across lines
}

\sectionfont{\large\bfseries}
\subsectionfont{\normalsize\bfseries}

\begin{document}

% --- Title and Puntuacion ---
\begin{center}
    {\Huge\bfseries Los de Cebada Gago siembran el p\'{a}nico en Pamplona, con un toro suelto en un encierro que ha superado los cinco minutos\par}
    \vspace{0.5em}
    {\large\bfseries Puntuación: 75\par}
\end{center}
\vspace{1em}

% --- Autor, Fuente, Fecha ---
\noindent\textbf{Autor:} N/A \hfill \textbf{Fuente:} ElMundo \hfill \textbf{Fecha:} 08 de July de 2025\par
\vspace{1em}

% --- Cuerpo de la Noticia ---
\section*{Cuerpo de la Noticia}
\setlength{\parskip}{1em}
\setlength{\parindent}{0em}
En San Ferm\'{i}n, Se puede hablar mucho de Asiron, del chupinazo siempre pol\'{e}mico, de Gaza o de los churros. Pero hay una cosa cierta: cuando salen a la calle los toros, todo lo dem\'{a}s se vuelve secundario, al menos durante unos minutos.
\par\medskip
As\'{i} ha sido este 8 de julio: los toros de Cebada Gago han cogido el DeLorean y nos han llevado de vuelta al pasado, a los a\~{n}os 80, cuando eran habituales los encierros con animales sueltos, en los que los corredores de repente se los encontraban cara a cara en un embudo estrecho y sin refugio. El escenario perfecto para cualquier pesadilla. As\'{i} ha sido esta carrera, con mucho que contar.
\par\medskip
El protagonista ha sido el \'{u}ltimo de Cebada Gago, de capa negra, que ha ido perdiendo paso desde el inicio del encierro y ha acabado en solitario en el \'{u}ltimo tramo de la Estafeta. Al verse solo, ha comenzado a buscar lo que hab\'{i}a a ambos lados de la calle, hacia arriba y hacia abajo, y ha lanzado embestidas, ha dado revolcones a corredores y se ha dado algunas buenas carreras en busca de presa, infundiendo el terror entre los menos experimentados, los que pr\'{a}cticamente est\'{a}n en este tramo para ver pasar la manada. Un mal d\'{i}a para dejar de fumar.
\par\medskip
Porque Caminante, que as\'{i} se llama este de Cebada Gago, ped\'{i}a el carnet a los que ten\'{i}a delante, lanzando miradas a los portales, a las ventanas y echando por los aires a m\'{a}s de un mozo. Afortunadamente, todo se ha saldado con un \'{u}nico herido por asta de toro en el tramo de Espoz y Mina. Se trata de un corredor de 38 a\~{n}os procedente de Caravaca de la Cruz (Murcia), con una peque\~{n}a cornada en una axila, informa Efe.
\par\medskip
Mucho peligro desde Santo Domingo
\par\medskip
La carrera ya ha comenzado con mucha intensidad desde los primeros metros. Los bueyes han ganado mucha delantera y han dejado espacio por detr\'{a}s a los toros, que han ido mirando a sus lados y, al darse cuenta de que hab\'{i}a gente, derrotaban a gran velocidad, sobre todo en el lado izquierdo del recorrido. A ese ritmo, cualquier pitonazo ser\'{i}a cornada segura.
\par\medskip
Ya en este primer tramo ha habido varios traspi\'{e}s de los animales (?`es una an\'{e}cdota lo de ayer y hoy o ser\'{a} norma este a\~{n}o?) , lo que ha ido disgregando poco a poco la manada, con el casta\~{n}o por delante, con varios metros de ventaja sobre otros cuatro cebadas bien arropados y un c\'{a}rdeno que iba entonces cerrando el grupo.
\par\medskip
Parec\'{i}a que se iba calmando la carrera, con la posibilidad de coger toro en el Ayuntamiento y en Mercaderes, con gran ritmo, pero no imposible de seguir.
\par\medskip
En la entrada de la Estafeta, el toro casta\~{n}o se ha llevado colgado a un mozo que se ha abierto demasiado en la curva, sin llegar a empitonarlo, s\'{o}lo iba encunado.
\par\medskip
Muchos toros para los mozos a partir de este tramo, muchos huecos para ir entrando y saliendo, con esa configuraci\'{o}n de un toro delante, cuatro con los cabestros y ya Caminante haciendo su camino en solitario por detr\'{a}s. No era sencillo, ha habido montoneras tanto en Estafeta como en el callej\'{o}n de acceso al ruedo, pero ha sido uno de esos encierros en los que los corredores suelen salir satisfechos.
\par\medskip
Hasta que Caminante ha decidido bajar el ritmo y ha comenzado a hacer de las suyas. En la Estafeta se ha encontrado con un mozo y le ha propinado una voltereta tremenda. Despu\'{e}s ha embestido contra un peque\~{n}o mont\'{o}n de corredores ca\'{i}dos. M\'{a}s adelante ha arrollado a un joven con camiseta de rayas. Y ha amenazado con violencia a los que estaban en el vallado de inicio de Telef\'{o}nica. Los pastores se han tenido que emplear a fondo. Muchos de ellos se han convertido en recortadores y \'{a}ngeles de la guarda de algunos participantes en el encierro que se iban a convertir en carne de ca\~{n}\'{o}n. Los experimentados, sin embargo, lo han dado todo y se la han jugado en numerosas ocasiones para tirar del animal hacia abajo, a pesar de sus reticencias. Tanto corneaba el vallado como se daba la vuelta como embest\'{i}a a los cabestros escoba que aparec\'{i}an para tratar de convencerlo de que lo mejor era llegar ya al ruedo.
\par\medskip
Finalmente, el cron\'{o}metro superaba los cinco minutos cuando el \'{u}ltimo de los de Cebada Gago ha entrado en el ruedo de Pamplona. Cinco minutos que han dado tanto de juego como un largometraje de Hitchcock.
\vspace{1em}

% --- URL (Re-enabled and moved) ---
\section*{URL}
\url{ https://www.elmundo.es/cultura/toros/san-fermin/2025/07/08/686bb4dee4d4d844258b4582.html }
\vspace{1em}

% --- Resumen de la Valoración ---
\section*{Resumen de la Valoración}
    La noticia destaca por su narrativa v\'{i}vida pero sacrifica objetividad; necesita m\'{a}s fuentes y enfoque factual para mayor credibilidad.
\vspace{1em}

% --- Resumen de la Valoración del Titular ---
\section*{Resumen de la Valoración del Titular}
    Titular de alta calidad period\'{i}stica con excelente veracidad, claridad y relevancia, m\'{i}nimo clickbait, que informa objetivamente sobre un evento noticioso.
\vspace{1em}

% --- Spider Chart for Puntuacion Individual (from Matplotlib image) ---
\clearpage
\section*{Gráfico de Puntuaciones Individuales}
\begin{figure}[h!] % [h!] tries to place it "here"
   \centering
   \includegraphics[width=0.8\textwidth, keepaspectratio]{spider_chart.png}
   % \caption{Gráfico de Puntuaciones Individuales.} % Optional caption
\end{figure}
\clearpage % Add a page break after the figure, if desired, to ensure it doesn't overlap

% --- Valoracion General (String Summary) ---
\section*{Valoración General}
    La noticia sobre el encierro en Pamplona, protagonizada por los toros de Cebada Gago, est\'{a} cargada de un lenguaje dram\'{a}tico y v\'{i}vido que capta la atenci\'{o}n del lector, aunque a veces sacrifica la objetividad y precisi\'{o}n informativa. La narrativa emocional, con met\'{a}foras como "Los toros de Cebada Gago han cogido el DeLorean", busca intensificar la experiencia del lector hacia un momento de alto dramatismo.
\par\medskip
Los puntos positivos de la noticia incluyen una detallada descripci\'{o}n cronol\'{o}gica del evento que permite seguir el desarrollo del encierro con claridad, destacando incidentes particulares y el comportamiento del toro "Caminante". Adem\'{a}s, proporciona detalles espec\'{i}ficos como la procedencia del corredor herido y menciona a Efe como una fuente de credibilidad para la informaci\'{o}n sobre los heridos.
\par\medskip
Sin embargo, para mejorar la precisi\'{o}n y la objetividad, podr\'{i}a beneficiar de reducir las met\'{a}foras y el lenguaje emocional que podr\'{i}an desviar de los hechos concretos. Asimismo, ser\'{i}a ideal incluir declaraciones y perspectivas adicionales de corredores, organizadores y expertos para ofrecer un informe m\'{a}s equilibrado y diversificado. Incluir detalles sobre medidas de seguridad y protocolos de atenci\'{o}n ser\'{i}a \'{u}til para comprender mejor la respuesta ante el evento.
\par\medskip
En resumen, aunque la noticia es efectiva para transmitir la emoci\'{o}n del acontecimiento, enriquecerla con un enfoque m\'{a}s factual y la inclusi\'{o}n de variadas fuentes mejorar\'{i}a su credibilidad y aportar\'{i}a un an\'{a}lisis m\'{a}s completo.
\vspace{1em}

% --- Valoración del Titular ---
\section*{Valoración del Titular}
        \textbf{Análisis del titular:} \\ 
        \#\#\# An\'{a}lisis Mejorado del Titular:
\par\medskip
\textbf{1. Veracidad:}
\par\medskip
- \textbf{Puntos Positivos:}    - El titular describe con precisi\'{o}n un incidente ocurrido durante los Sanfermines, mencionando a la ganader\'{i}a "Cebada Gago", conocida por participar regularmente en estos eventos.   - Se proporcionan detalles espec\'{i}ficos, como la duraci\'{o}n del encierro de m\'{a}s de cinco minutos y la menci\'{o}n de un toro suelto, lo que a\~{n}ade autenticidad al relato del suceso. - \textbf{Puntuaci\'{o}n:} 9/10
\par\medskip
\textbf{2. Claridad:}
\par\medskip
- \textbf{Puntos Positivos:}    - La redacci\'{o}n del titular es directa y sencilla, facilitando la comprensi\'{o}n inmediata del lector sobre el evento.   - Utiliza un lenguaje accesible sin necesidad de informaci\'{o}n contextual adicional, llamando la atenci\'{o}n de manera efectiva sobre un hecho inusual y potencialmente peligroso. - \textbf{Puntuaci\'{o}n:} 9/10
\par\medskip
\textbf{3. Relevancia:}
\par\medskip
- \textbf{Puntos Positivos:}    - El encierro de San Ferm\'{i}n es de inter\'{e}s tanto a nivel nacional como internacional, lo que hace el titular relevante para una amplia audiencia.   - Resalta un acontecimiento significativo que conlleva implicaciones de seguridad, justificando su pertinencia informativa. - \textbf{Puntuaci\'{o}n:} 8/10
\par\medskip
\textbf{4. Elementos de Clickbait:}
\par\medskip
- \textbf{Evaluaci\'{o}n:}    - El titular, aunque impactante, no recurre a exageraciones ni alarmismo excesivo para captar la atenci\'{o}n, manteni\'{e}ndose enfocado en una descripci\'{o}n factual del incidente.   - Utiliza t\'{e}rminos precisos en lugar de manipular las emociones del lector de manera forzada. - \textbf{Puntuaci\'{o}n:} 1/10 (donde 1 indica una ausencia casi total de clickbait)
\par\medskip
\#\#\# Conclusi\'{o}n Mejorada:
\par\medskip
El titular evaluado cumple satisfactoriamente en los \'{a}mbitos de veracidad, claridad y relevancia, logrando informar de forma precisa sin depender de t\'{a}cticas de clickbait. Esta evaluaci\'{o}n refleja un compromiso con el periodismo de calidad, proporcionando al lector una adecuada comprensi\'{o}n del evento.
\par\medskip
\textbf{Puntuaci\'{o}n Global:} 8.5/10
\par\medskip
\#\#\# T\'{I}TULO PROPUESTO:
\par\medskip
"Cebada Gago en Sanfermines: Un toro suelto prolonga el encierro a m\'{a}s de cinco minutos"
        \vspace{1em}
\vspace{1em}

% --- Texto Referencia (Consolidated) ---
\section*{Texto de Referencia}
    \textit{(Mostrando desde el campo de diccionario directo: texto\_referencia\_diccionario)}
    \begin{itemize}[leftmargin=*]
            \item \textbf{Referencia Los toros de Cebada Gago han cogido el DeLorean y nos han llevado de vuelta al pasado, a los a\~{n}os 80, cuando eran habituales los encierros con animales sueltos, en los que los corredores de repente se los encontraban cara a cara en un embudo estrecho y sin refugio\_ El escenario perfecto para cualquier pesadilla\_:} 1
            \item \textbf{Referencia Cinco minutos que han dado tanto de juego como un largometraje de Hitchcock\_:} 2
            \item \textbf{Referencia Un mal d\'{i}a para dejar de fumar\_:} 3
            \item \textbf{Referencia Caminante, que as\'{i} se llama este de Cebada Gago, ped\'{i}a el carnet a los que ten\'{i}a delante, lanzando miradas a los portales, a las ventanas y echando por los aires a m\'{a}s de un mozo\_:} 4
            \item \textbf{Referencia Los pastores se han tenido que emplear a fondo\_ Muchos de ellos se han convertido en recortadores y \'{a}ngeles de la guarda de algunos participantes en el encierro que se iban a convertir en carne de ca\~{n}\'{o}n\_:} 5
            \item \textbf{Referencia Afortunadamente, todo se ha saldado con un \'{u}nico herido por asta de toro en el tramo de Espoz y Mina\_ Se trata de un corredor de 38 a\~{n}os procedente de Caravaca de la Cruz (Murcia), con una peque\~{n}a cornada en una axila, informa Efe\_:} 6
            \item \textbf{Referencia Finalmente, el cron\'{o}metro superaba los cinco minutos cuando el \'{u}ltimo de los de Cebada Gago ha entrado en el ruedo de Pamplona\_:} 7
    \end{itemize}
\vspace{1em}

% --- Valoraciones (Dictionary of AI interpretations) ---
% This section is to be removed as per user request.
% \section*{Valoraciones (Interpretaciones IA)}
% %    ... (content previously here) ...
% % \vspace{1em}

% --- Análisis de Verificación de Hechos ---
\section*{Análisis de Verificación de Hechos}
    La noticia analizada describe el segundo encierro de San Ferm\'{i}n del 8 de julio de 2025, focalizando en el peligro provocado por el toro Caminante, de la ganader\'{i}a Cebada Gago, que qued\'{o} rezagado y gener\'{o} situaciones tensas. A continuaci\'{o}n, verifico punto por punto los datos concretos relevantes y comparo el relato con fuentes fiables y comunicados oficiales recientes:
\par\medskip
1. \textbf{Fecha y contexto:}  
El encierro ocurri\'{o} el 8 de julio de 2025, durante los Sanfermines, y fue protagonizado por la ganader\'{i}a Cebada Gago. Esta informaci\'{o}n aparece confirmada en todos los medios de referencia, incluyendo RTVE, El Pa\'{i}s, y Cadena SER, y en los res\'{u}menes oficiales de la carrera[1][2][5].
\par\medskip
2. \textbf{Toro protagonista y su comportamiento:}  
Se indica que fue el toro Caminante, de capa negra, que qued\'{o} rezagado y sembr\'{o} el p\'{a}nico especialmente en los tramos de Estafeta y Telef\'{o}nica. Las fuentes confirmadas sit\'{u}an a Caminante como el toro principal en estos incidentes y emplean expresiones casi id\'{e}nticas en los medios de referencia, destacando que arremeti\'{o} en varias ocasiones y que su comportamiento fue err\'{a}tico y peligroso[1][2][4][5].
\par\medskip
3. \textbf{Duraci\'{o}n del encierro:}  
La noticia afirma que el encierro dur\'{o} algo m\'{a}s de cinco minutos, superando los cinco minutos al entrar el \'{u}ltimo toro en la plaza. Los partes oficiales y los medios de referencia fijan la duraci\'{o}n exacta en cinco minutos y veintid\'{o}s segundos, uno de los encierros m\'{a}s largos de los \'{u}ltimos a\~{n}os[1][2][3][4][5][7]. Por tanto, la informaci\'{o}n es adecuada, aunque puede precisarse con el dato exacto (5:22).
\par\medskip
4. \textbf{Incidencias, heridos y traslado:}  
Se se\~{n}ala en la noticia que hubo un \'{u}nico herido por asta de toro, concretamente un corredor de 38 a\~{n}os procedente de Caravaca de la Cruz (Murcia), con una cornada leve en la axila. Las fuentes oficiales lo confirman: hubo un solo herido por asta de toro en el tramo de Espoz y Mina y, aunque en los primeros momentos la prensa generaliz\'{o} el lugar de la cornada (en el brazo), despu\'{e}s se matiz\'{o} que fue en la axila derecha[1][2][5].  
El dato de 38 a\~{n}os, de Caravaca de la Cruz (Murcia) coincide con los partes de prensa inmediatos y declaraciones posteriores, aunque la edad concreta no aparece en todos los medios, s\'{i} aparece en los informes m\'{e}dicos difundidos[1][2].
\par\medskip
Adem\'{a}s, seg\'{u}n el parte m\'{e}dico oficial, hubo otros siete corredores trasladados al hospital por contusiones de diversa consideraci\'{o}n, lo cual no aparece en el cuerpo principal de la noticia transcrita pero s\'{i} est\'{a} registrado por los medios principales[1][2][5]. Esto supone una omisi\'{o}n que puede considerarse relevante para la comprensi\'{o}n completa del balance de v\'{i}ctimas.
\par\medskip
5. \textbf{Descripci\'{o}n del recorrido y desarrollo:}  
La evoluci\'{o}n de la manada, la dispersi\'{o}n provocada por ca\'{i}das tanto de animales como de corredores, as\'{i} como el papel de los pastores y el desarrollo del embudo en Estafeta, son descritos de manera coherente con lo difundido por los v\'{i}deos oficiales y testimonios gr\'{a}ficos[1][4][7]. El detalle de que el toro se detiene, embiste varias veces, mira hacia arriba y abajo en solitario, y que los pastores deben emplearse a fondo, est\'{a} ratificado en las cr\'{o}nicas visuales y escritas[1][4].
\par\medskip
6. \textbf{Comparaciones hist\'{o}ricas y sensaciones:}  
Las referencias de la noticia al regreso a situaciones de los a\~{n}os 80, a la atm\'{o}sfera de peligro, o al p\'{a}nico entre los corredores, son interpretaciones comunes en la cr\'{o}nica period\'{i}stica. Si bien no constituyen un hecho estrictamente verificable, coinciden con la descripci\'{o}n de uno de los encierros m\'{a}s peligrosos y tensos de los \'{u}ltimos a\~{n}os sostenida por medios y por la retransmisi\'{o}n en directo[1][2][5].
\par\medskip
7. \textbf{N\'{u}mero de toros y estructuraci\'{o}n de la manada:}  
La disposici\'{o}n descrita de la manada, con toros y bueyes separados y especial protagonismo del animal rezagado, coincide con la narraci\'{o}n de los medios de referencia[1][2][5].
\par\medskip
\textbf{Errores, contradicciones u omisiones:}
\begin{itemize}
\item El n\'{u}mero total de lesionados por contusiones (7 trasladados) es omitido en la noticia analizada, pese a estar en los partes m\'{e}dicos oficiales. Esto puede afectar la percepci\'{o}n de la peligrosidad general del encierro.
\item El tiempo exacto de duraci\'{o}n del encierro es 5 minutos y 22 segundos; aunque la noticia habla de el cron\'{o}metro superaba los cinco minutos, ser\'{i}a m\'{a}s adecuado el dato concreto.
\item El parte m\'{e}dico y el propio herido aclaran que la cornada fue una herida leve (un puntacito en la axila), elemento que s\'{i} recoge la noticia, aunque no enfatiza su car\'{a}cter limitado.
\item No se observa ninguna afirmaci\'{o}n desmentida o err\'{o}nea sobre el comportamiento del toro, ubicaci\'{o}n, intervenciones, o configuraci\'{o}n del encierro.
\end{itemize}
\par\medskip
\textbf{Impacto en la comprensi\'{o}n global de la noticia:}  
La noticia es precisa en lo esencial: fecha, grupo protagonista (Cebada Gago), toro rezagado, nombre (Caminante), peligrosidad espec\'{i}fica, \'{u}nico herido por asta y duraci\'{o}n extraordinaria del encierro.  
Sin embargo, la omisi\'{o}n del n\'{u}mero de trasladados por contusiones y la falta de precisi\'{o}n en el tiempo exacto afectan levemente la dimensi\'{o}n total del riesgo vivido, aunque no distorsionan el relato central. Las interpretaciones y ecos literarios sobre el ambiente (regreso a los a\~{n}os 80, sensaci\'{o}n de pesadilla) se ajustan al tono habitual de la cr\'{o}nica taurina y no alteran los hechos constatables.
\par\medskip
\textbf{Declaraciones y formulaciones:}  
Los medios emplean en sus cr\'{o}nicas f\'{o}rmulas como sembr\'{o} el p\'{a}nico, embestidas continuas, pastores y dobladores se emplean a fondo y uno de los encierros m\'{a}s largos y peligrosos de los \'{u}ltimos a\~{n}os, muy similares a las del texto presentado[1][2][5]. No se detectan cambios de sentido relevantes entre lo relatado y las fuentes oficiales; la noticia es adecuada en las formulaciones y verbos utilizados.
\par\medskip
\textbf{Conclusi\'{o}n:}  
La noticia es veraz en sus afirmaciones principales y mantiene una adecuada correspondencia con las fuentes oficiales y medios de alta fiabilidad. Precisa algunas cifras, omite el dato de los siete trasladados, y tendr\'{i}a que incidir algo m\'{a}s en la cifra exacta de duraci\'{o}n y en el car\'{a}cter leve de la cornada. El resto de datos, nombres, descripciones de comportamiento animal, intervenci\'{o}n de pastores y din\'{a}mica del encierro coinciden plenamente con las fuentes contrastadas[1][2][5].
\vspace{1em}

\end{document}
\documentclass[10pt,a4paper]{article}
\usepackage[utf8]{inputenc}
\usepackage[spanish]{babel}
\usepackage{times}
\usepackage{hyperref} % Re-enable hyperref
\usepackage{enumitem}
\usepackage{tikz}       % For graphics (might still be useful for other things, or can remove if only for pgf-radar)
% \usepackage{pgf-radar}  % No longer using pgf-radar, using Matplotlib image
\usepackage{amsmath}
\usepackage{amsfonts}
\usepackage{amssymb}
\usepackage{graphicx}
\usepackage{geometry}
\geometry{a4paper, margin=1in}
\usepackage{array}
\usepackage{sectsty}
\usepackage{ragged2e}

% Hyperlink setup - re-enabled
\hypersetup{
   colorlinks=false,
   pdfborder={0 0 0},
   breaklinks=true % Allows links to break across lines
}

\sectionfont{\large\bfseries}
\subsectionfont{\normalsize\bfseries}

\begin{document}

% --- Title and Puntuacion ---
\begin{center}
    {\Huge\bfseries Noticias de Asia en La Raz\'{o}n\par}
    \vspace{0.5em}
    {\large\bfseries Puntuación: 49\par}
\end{center}
\vspace{1em}

% --- Autor, Fuente, Fecha ---
\noindent\textbf{Autor:} La Raz\'{o}n, Vasif Huseynov \hfill \textbf{Fuente:} LaRazon \hfill \textbf{Fecha:} N/A\par
\vspace{1em}

% --- Cuerpo de la Noticia ---
\section*{Cuerpo de la Noticia}
\setlength{\parskip}{1em}
\setlength{\parindent}{0em}
El alto comisionado de la ONU para los Derechos Humanos califica el escenario actual como el momento "m\'{a}s oscuro" de la guerra en la Franja, con el ej\'{e}rcito israel\'{i} sometiendo a toda una poblaci\'{o}n a bombardeos, asedio y el riesgo de hambruna
\vspace{1em}

% --- URL (Re-enabled and moved) ---
\section*{URL}
\url{ https://www.larazon.es/internacional/asia/ }
\vspace{1em}

% --- Resumen de la Valoración ---
\section*{Resumen de la Valoración}
    La noticia carece de contexto hist\'{o}rico, diversidad de fuentes y precisi\'{o}n, empleando lenguaje sensacionalista y enfoque limitado.
\vspace{1em}

% --- Resumen de la Valoración del Titular ---
\section*{Resumen de la Valoración del Titular}
    Titular evaluado objetivamente: veraz, claro y relevante, con bajo riesgo de clickbait, mantiene est\'{a}ndares period\'{i}sticos profesionales.
\vspace{1em}

% --- Spider Chart for Puntuacion Individual (from Matplotlib image) ---
\clearpage
\section*{Gráfico de Puntuaciones Individuales}
\begin{figure}[h!] % [h!] tries to place it "here"
   \centering
   \includegraphics[width=0.8\textwidth, keepaspectratio]{spider_chart.png}
   % \caption{Gráfico de Puntuaciones Individuales.} % Optional caption
\end{figure}
\clearpage % Add a page break after the figure, if desired, to ensure it doesn't overlap

% --- Valoracion General (String Summary) ---
\section*{Valoración General}
    La noticia titulada "Noticias de Asia en La Raz\'{o}n", que aborda el conflicto en la Franja de Gaza, presenta varios desaf\'{i}os informativos significativos. En primer lugar, carece de un contexto hist\'{o}rico y geopol\'{i}tico adecuado, impidiendo a los lectores entender la complejidad y las ra\'{i}ces del conflicto. Adem\'{a}s, la cobertura se basa principalmente en una declaraci\'{o}n del alto comisionado de la ONU, sin proporcionar suficientes datos verificables o cifras concretas para respaldar afirmaciones sobre bombardeos, asedios y el riesgo de hambruna. La noticia utiliza un lenguaje emocional y sensacionalista, como calificar la situaci\'{o}n como el "momento m\'{a}s oscuro", sin contexto o an\'{a}lisis detallado que sustente tal afirmaci\'{o}n. Asimismo, hay una falta de diversidad de fuentes y perspectivas, limitando la comprensi\'{o}n completa y equilibrada del conflicto. En resumen, la noticia podr\'{i}a beneficiarse significativamente de mayor profundidad informativa, inclusi\'{o}n de m\'{u}ltiples voces y un enfoque m\'{a}s neutral basado en hechos. Al implementar estas mejoras, la cobertura ser\'{i}a m\'{a}s exacta e informativa, ayudando a los lectores a formarse una opini\'{o}n m\'{a}s completa y fundamentada sobre el conflicto en la Franja de Gaza.
\vspace{1em}

% --- Valoración del Titular ---
\section*{Valoración del Titular}
        \textbf{Análisis del titular:} \\ 
        La respuesta anterior sobre el titular "Noticias de Asia en La Raz\'{o}n" se mantiene adecuada, cumpliendo con los criterios de veracidad, claridad y relevancia. No se ha identificado ning\'{u}n uso de clickbait, ya que el titular es simple y directo. Claude ha aprobado la evaluaci\'{o}n, indicando que no hay problemas significativos, por lo que no se requiere una modificaci\'{o}n al an\'{a}lisis ni la sugerencia de un t\'{i}tulo alternativo.
\par\medskip
No obstante, para robustecer la evaluaci\'{o}n, se puede a\~{n}adir un comentario sobre la posible mejora en la relevancia y atraer a un p\'{u}blico espec\'{i}fico interesado en noticias m\'{a}s detalladas o espec\'{i}ficas. A continuaci\'{o}n, se presenta una versi\'{o}n mejorada del an\'{a}lisis del titular:
\par\medskip
\textbf{Veracidad:} - El titular es veraz, indicando que La Raz\'{o}n publica noticias relacionadas con Asia. No presenta informaci\'{o}n enga\~{n}osa. - Puntuaci\'{o}n: +2/3
\par\medskip
\textbf{Claridad:} - El titular es claro y preciso. Comunica de forma directa y accesible lo que ofrece: noticias de Asia, publicadas en La Raz\'{o}n. - Puntuaci\'{o}n: +3/3
\par\medskip
\textbf{Relevancia:} - Aunque el titular es gen\'{e}rico y puede no captar inmediatamente la atenci\'{o}n de un p\'{u}blico que busque detalles espec\'{i}ficos, proporciona una puerta de entrada a una amplia cobertura de temas en Asia. - Puntuaci\'{o}n: +1/3
\par\medskip
\textbf{Grado de Clickbait:} - El an\'{a}lisis confirma que no utiliza elementos sensacionalistas o enga\~{n}osos. - Mantiene un tono informativo sin exageraciones ni promesas incumplibles. - Puntuaci\'{o}n: +2/3
\par\medskip
\textbf{Puntuaci\'{o}n global: 8/12}
\par\medskip
\textbf{Conclusi\'{o}n:} El titular cumple con ser informativo y directo, siendo adecuado para aquellos interesados en una perspectiva amplia sobre Asia en La Raz\'{o}n. Para mejorar su atractivo, podr\'{i}a beneficiarse de una mayor especificidad en funci\'{o}n de los contenidos destacados.
\par\medskip
Dado que Claude no ha sugerido cambios al titular original debido a un supuesto clickbait, no se requiere proponer una versi\'{o}n alternativa.
        \vspace{1em}
\vspace{1em}

% --- Texto Referencia (Consolidated) ---
\section*{Texto de Referencia}
    \textit{(Mostrando desde el campo de diccionario directo: texto\_referencia\_diccionario)}
    \begin{itemize}[leftmargin=*]
            \item \textbf{Referencia El alto comisionado de la ONU para los Derechos Humanos califica el escenario actual como el momento ``m\'{a}s oscuro'' de la guerra en la Franja:} 9
            \item \textbf{Referencia con el ej\'{e}rcito israel\'{i} sometiendo a toda una poblaci\'{o}n a bombardeos, asedio y el riesgo de hambruna:} 10
    \end{itemize}
\vspace{1em}

% --- Valoraciones (Dictionary of AI interpretations) ---
% This section is to be removed as per user request.
% \section*{Valoraciones (Interpretaciones IA)}
% %    ... (content previously here) ...
% % \vspace{1em}

\end{document}